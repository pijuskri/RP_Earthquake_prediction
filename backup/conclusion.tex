\section{Conclusion and future work}
This study analyzed the question of if shallow or deep earthquakes are better suited for short-term deep learning prediction. Data was used and pre-processed from GeoNet, covering stations in New Zealand. Two data sets were created, separating deep and shallow earthquakes in each, split along the depth of 70km. An LSTM model took 30 seconds of input and tried to predict the occurrence of an earthquake 3 seconds in the future. Each model used an equal number of active samples, with an equal of number of background, non-active samples. The models were evaluated by training them 10 times and comparing the final mean test accuracy, recall and precision. The results showed a very similar accuracy for both models, with an accuracy of  0.869 for the shallow one, compared to 0.850 of the deep one. Thus, it can be concluded that deep learning techniques are similarly capable of predicting both shallow and deep earthquakes. \\
\\
The results presented do have limitations for general conclusions for all cases. Most importantly, data from different sources and regions should also be considered, as different earthquake prone regions contain possibly unique types of tectonic processes. Further, the model used could also potentially have an effect on how different data is treated, thus the method presented could be extended for use for many models. Finally, data pre-processing is a complex matter in the field of signals, especially natural ones. Different steps taken and values used in down sampling, normalization and could introduce more significant differences in the signal used for final training. Finally, the preprocessing steps taken regarding station filtering in Section 3.1.2 were done using a draft model by comparing accuracies achieved. Therefore, this processing step is model dependent, and depending on hyper parameters and specific deep learning model used, the final station list could be different. 